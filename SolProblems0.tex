\begin{solution}[1.1]{Kempen, S.F.M.} 
    
\noindent a) To be proven: $A\cap (B\cup C) = (A\cap B) \cup (A\cap C)$.

\noindent $``\subseteq"$ Let $x \in A\cap(B\cup C)$ then $x\in A$ and $x\in B\cup C$, which means $x\in A$ and ($x\in B$ or $x\in C$). 
\begin{itemize}
	\item If $x\in A$ and $x\in B$, then $x\in A\cap B$ so also $x\in (A\cap B) \cup (A\cap C)$.
	\item If $x\in A$ and $x\in C$, then $x\in A\cap C$ so also $x\in (A\cap B) \cup (A\cap C)$.
\end{itemize}

\noindent $``\supseteq"$ Let $x\in (A\cap B) \cup (A\cap C)$ then $x\in A\cap B$ or $x\in A\cap C$.
\begin{itemize}
	\item If $x\in A\cap B$, then $x\in A$ and $x\in B$ so $x\in B\cup C$ so also $x\in A\cap (B \cup C)$.
	\item If $x\in A\cap C$, then $x\in A$ and $x\in C$ so $x\in B\cup C$ so also $x\in A\cap (B \cup C)$.
\end{itemize}

\noindent b) To be proven: $A\cup(B\cap C) = (A\cup B) \cap (A\cup C)$.

\noindent $``\subseteq"$ Let $x\in A\cup (B\cap C)$ then $x\in A$ or $x\in B\cap C$, which means $x\in A$ or ($x\in B$ and $x\in C$). 
\begin{itemize}
	\item If $x\in A$, then $x\in A\cup B$ and $x\in A\cup C$ so also $x\in (A\cup B) \cap (A\cup C)$.
	\item If $x\in B$ and $x\in C$, then $x\in A\cup B$ and $x\in A\cup C$ so also $x\in (A\cup B) \cap (A\cup C)$.
\end{itemize}
\noindent $``\supseteq"$ Let $x\in (A\cup B) \cap (A\cup C)$ then $x\in A\cup B$ and $x\in A\cup C$, which means $(x\in A$ or $x\in B$) and ($x\in A$ or $x\in C$). 
\begin{itemize}
	\item If $x\in A$ then definitely $x\in A\cup (B\cap C)$.
	\item If $x\notin A$ then $x\in B$ and $x\in C$ which means $x\in B\cap C$ so also $x\in A\cup (B\cap C)$.
\end{itemize}
\end{solution}


\begin{solution}[1.2]{Kempen, S.F.M.} To be proven: $A\cap (A\cup B) = A$. 

\noindent $``\subseteq"$ Let $x\in A\cap (A\cup B)$, then $x\in A$ so we are done.

\noindent $``\supseteq"$ Let $x\in A$, then also ($x\in A$ or $x\in B$) is true, therefore $x\in A\cup B$. So $x\in A\cap (A\cup B)$. 
\end{solution}

\begin{solution}[1.5]{Kempen, S.F.M.} To be proven: $\Omega \setminus \left(\bigcup_{i\in I} A_i \right) = \bigcap_{i\in I} \Omega \setminus A_i$. 

\noindent $``\subseteq"$ Let $x\in \Omega \setminus \left(\bigcup_{i\in I} A_i \right)$ then $x\in \Omega$ and $x\notin \bigcup_{i\in I} A_i$, so for all $i \in I $ holds $x\notin A_i$. Then for all $i \in I$ we have $x\in \Omega\setminus A_i$. Since this is true for any $i\in I$, we can write $x\in \bigcap_{i\in I} \Omega \setminus A_i$. 

\noindent $``\supseteq"$ Let $x\in \bigcap_{i\in I} \Omega \setminus A_i$ then for all $i\in I$ we have $x\in \Omega\setminus A_i$, so $x\in \Omega$ and $x\notin A_i$. Since this holds for all $i\in I$, we can write $x\notin \bigcup_{i\in I} A_i $ and therefore $x\in \Omega \setminus \left(\bigcup_{i\in I} A_i \right)$.
\end{solution}

\begin{solution}[1.8]{Kempen, S.F.M.} 
    
\noindent a) The statement $f^{-1}(f(A)) = A$ is not true since $f$ is not assumed to be injective. As a counterexample, take $\Omega = \{0,1\}, E = \{0\}, f(\{0\}) = f(\{1\}) = \{0\}, A = \{0\}$ then $f(A) = \{0\}$ and $f^{-1}(f(A)) = f^{-1}(\{0\}) = \{0,1\} \neq A$.

\noindent b) The statement $f(f^{-1}(H)) = H$ is not true since f is not assumed to be surjective. As a counterexample, take $\Omega = \{1\}$, $E = \{1,2\}$, $f(\{1\}) = \{1\}$, $H = \{1,2\}$ then $f^{-1}(H) = f^{-1}(\{1,2\}) = \{1\}$ and $f(f^{-1}(H)) = f(\{1\}) = \{1\} \neq H$.
\end{solution}

\begin{solution}[1.10]{Beurskens, T.P.J.}
Let $n \in \bbN$, and define $K = \{1, \ldots, n\}$.
By definition of the supremum, we then have
$$\sum_{i = 1}^n a_i = \sum_{i \in K} a_i \leq \sup \left( \sum_{i \in K} a_i : K \subseteq I, K~\text{finite} \right) = \sum_{i \in I} a_i.$$
Letting $n \to \infty$, we get
\[
\lim_{n \to \infty} \sum^n_{i = 1} a_i \leq \sum_{i \in I} a_i.
\]
Next, let $K \subseteq \bbN$ be finite, so that $K \subset \{1, \ldots, n\}$ for some $n \in \bbN$.
Note that $n \geq \sup K$.
We get
\[
\sum_{i \in K} a_i \leq \sum_{i \in \{1,\ldots, n\}} a_i = \sum_{i = 1}^n  a_i \leq \lim_{n\to\infty} \sum_{i = 1}^n  a_i.
\]
Since this holds for arbitrary finite $K$, it holds for all finite $K$.
Thus we get
\[
\sum_{i \in I} a_i = \sup \left( \sum_{i \in K} a_i : K \subseteq I, K~\text{finite} \right) \leq \lim_{n\to\infty} \sum_{i = 1}^n  a_i.
\]
Using both inequalities, we see that indeed $\sum_{i \in I} a_i = \lim_{n \to \infty} \sum^n_{i = 1} a_i$. 

We are left with showing that the sum $\sum_{i = 1}^\infty a_i$ does not depend on the ordering of the elements in the sequence $(a_i)$. This follows immediately from the fact that for any index set $I$
\begin{equation*}
	\sum_{i \in I} a_i = \sup\Big\{\sum_{i\in K} a_i:\, K\subseteq I,\text{ $K$ finite}\Big\}
\end{equation*}
is completely blind to any ordering of $I$, in fact $I$ is possibly not even ordered. To be more precise, if $\sigma:I\to I$ is a bijection, then
\begin{equation*}
	\begin{aligned}
	\sum_{i \in I} a_{\sigma(i)} & =  \sup\Big\{\sum_{i\in K} a_{\sigma(i)}:\, K\subseteq I,\text{ $K$ finite}\Big\} \\ 
	& = \sup\Big\{\sum_{i\in \sigma^{-1}(K)} a_i:\, \sigma^{-1}(K)\subseteq I,\text{ $\sigma^{-1}(K)$ finite}\Big\} = \sum_{i \in I} a_{i},
	\end{aligned}
\end{equation*}
where we used that $K\subseteq I$ is finite if and only if $\sigma^{-1}(K)$ is finite.
So, from this observation and the first part of the problem, one has that for any bijection $\sigma:\bbN\to \bbN$
\begin{equation*}
	\sum^\infty_{i = 1} a_i = \sum_{i\in \bbN} a_i = \sum_{i\in \bbN} a_{\sigma(i)} = \sum_{i = 1}^\infty a_{\sigma(i)},
\end{equation*}
where the summation in the middle is with respect the new notion.
\end{solution}

\begin{solution}[1.12]{Bakker, A.}
	Proof by contradiction. Suppose $\sum_{i\in I}a_i < \infty$, then by Problem 1.11 we have that the set $I$ contains at most a countable number of elements $i$ with $a_i$ positive. This, together with the fact that $a_i>0$ for all $i\in I$, contradicts that there are uncountable many elements in $I$. Hence  $\sum_{i\in I}a_i = \infty$.
\end{solution}