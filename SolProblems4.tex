\begin{solution}[5.3]{Kempen, S.F.M.}

    \noindent "$\Rightarrow$" Let $s,t\in \mathbb{R}$ and define $A = \{X \geq s\}, B = \{Y \geq t\}$. We have
    $$ A = \{X \geq s\} = \{\omega\in \Omega: X(\omega) \geq s\} = X^{-1}([s,\infty)). $$
    By definition $\sigma(X)$ is the smallest $\sigma$-algebra such that $X^{-1}(B)\in \sigma(X)$ for all $B\in \mathcal{B}_\mathbb{R}$ and $[s,\infty)\in \mathcal{B}_\mathbb{R}$ so surely $A\in \sigma(X)$. In a similar way it can be shown that $B\in \sigma(Y)$. Now
    $$ \mathbb{P}(X\geq s, Y\geq t) = \mathbb{P}(\{X\geq s\} \cap \{Y\geq t\}) = \mathbb{P}(A\cap B) $$
    and by our assumption
    $$ \mathbb{P}(A\cap B)  = \mathbb{P}(A) \mathbb{P}(B) = \mathbb{P}(\{X\geq s\})\cdot\mathbb{P}(\{Y\geq t\}) = \mathbb{P}(X\geq s)\cdot\mathbb{P}(Y\geq t). $$
    
    
    \noindent "$\Leftarrow$" For the first part of the proof, let us define
    \begin{align*}
        \Lambda_s &= \{ \ B\in \sigma(Y): \mathbb{P}(X\geq s, B) = \mathbb{P}(X\geq s)\mathbb{P}(B)\ \}.
    \end{align*}
    $\Lambda_s$ is a collection of sets in $\sigma(Y)$ so $\Lambda_s \subseteq \sigma(Y)$. Furthermore we can show that $\Lambda_s$ is a $\lambda$-system.
    \begin{enumerate}
        \item We have $\Omega \in \Lambda_s$ since 
        \begin{align*}
            \mathbb{P}(X\geq s, \Omega) &= \mathbb{P}(\{X\geq s\} \cap \Omega) = \mathbb{P}(\{\omega\in \Omega: X(\omega) \geq s\}\cap \Omega) \\
            &= \mathbb{P}(\{\omega\in \Omega: X(\omega) \geq s\}) \\
            &= \mathbb{P}(X\geq s) \cdot 1 = \mathbb{P}(X\geq s)\mathbb{P}(\Omega).
        \end{align*}
        \item Let $A \in \Lambda_s$ then 
        \begin{align*}
            \mathbb{P}(X\geq s, \Omega\setminus A) &= \mathbb{P}(\{X\geq s\} \cap \Omega\setminus A) \\
            &= \mathbb{P}(\{\omega\in \Omega: X(\omega)\geq s\} \cap \Omega\setminus A)\\
            &= \mathbb{P}(\{\omega\in \Omega\setminus A: X(\omega)\geq s\})\\
            &= \mathbb{P}(X\geq s) - \mathbb{P}(\{X\geq s\} \cap A)\\
            &= \mathbb{P}(X\geq s)(1-\mathbb{P}(A)) \quad (\textrm{since }A\in\Lambda_s)\\
            &= \mathbb{P}(X\geq s)\mathbb{P}(\Omega\setminus A). 
        \end{align*}
        \item Let $(A_i) \subset \Lambda_s$ be a sequence of mutually disjoint sets then by  the $\sigma$-additivity of the measure we have
        \begin{align*}
            \mathbb{P}(X\geq s, \bigcup_{i=1}^\infty A_i) &= \mathbb{P}(\{X\geq s\}\cap \bigcup_{i=1}^\infty A_i)\\
            &= \mathbb{P}(\{\omega\in \left(\Omega\cap \bigcup_{i=1}^\infty A_i\right): X(\omega)\geq s\} )\\
            &= \sum_{i=1}^\infty \mathbb{P}( \{\omega\in \Omega\cap A_i: X(\omega)\geq s\} )\\
            &= \sum_{i=1}^\infty \mathbb{P}(\{X\geq s\} \cap A_i)\\
            &= \sum_{i=1}^\infty \mathbb{P}(\{X\geq s\}) \mathbb{P}(A_i)\\
            &= \mathbb{P}(X\geq s) \mathbb{P}(\bigcup_{i=1}^\infty A_i).
        \end{align*}
    \end{enumerate}
    So we conclude that $\Lambda_s$ is a $\lambda$-system. By assumption, we also have $\{Y\geq t\}\subseteq \Lambda_s$ for all $t\in \mathbb{R}$.
    
    \noindent Next we define 
    $$ \mathcal{E} = \{ \{Y\geq t\},t\in \mathbb{R}\}$$
    and show that this is a $\pi$-system. Let $s,r\in \mathbb{R}$ then $\{Y\geq s\},\{Y\geq r\}\in \mathcal{E}$ and
    $$ \{Y\geq s\} \cap \{Y\geq r\} = \{Y\geq \min(s,r)\} \in \mathcal{E},$$
    since the minimum of two real numbers is just a real number. Thus $\mathcal{E}$ is a $\pi$-system.
    
    \noindent Note $\sigma(Y) \subseteq \sigma(\mathcal{E})$ by definition of $\sigma(Y)$ and next to that $\mathcal{E} \subseteq \Lambda_s$ by our assumption. We can now apply the $\pi$-$\lambda$ theorem, which tells us
    $$ \sigma(Y) \subseteq \sigma(\mathcal{E}) \stackrel{\pi-\lambda}{\subseteq} \Lambda_s \subseteq \sigma(Y), $$
    so in fact we have $\Lambda_s = \sigma(Y)$. Thus we can conclude that $\mathbb{P}(X\geq s, B) = \mathbb{P}(X\geq s)\mathbb{P}(B)$ holds for all $B\in \sigma(Y)$.
    
    \noindent For the second part of the proof, define
    $$ \mathcal{H} = \{A\in \sigma(X): \mathbb{P}(A\cap B) = \mathbb{P}(A)\mathbb{P}(B) \ \forall  B\in \sigma(Y) \}. $$
    We will show that $\mathcal{H}$ is a $\lambda$-system. Let $B\in \sigma(Y)$.
    \begin{enumerate}
        \item We have $\Omega \in \mathcal{H}$ since $\mathbb{P}(\Omega\cap B) = \mathbb{P}(B) = \mathbb{P}(B)\mathbb{P}(\Omega)$.
        \item Let $A \in \mathcal{H}$ then 
        \begin{align*}
            \mathbb{P}(B \cap (\Omega\setminus A) ) &= \mathbb{P}(B\setminus (A\cap B))\\
            &= \mathbb{P}(B) - \mathbb{P}(A\cap B) \quad \textrm{(by exclusion)}\\
            &= \mathbb{P}(B)(1-\mathbb{P}(A)) \quad \textrm{(since } A\in \mathcal{H})\\
            &= \mathbb{P}(B)\mathbb{P}(\Omega \setminus A).
        \end{align*}
        \item Let $(A_i) \subset \mathcal{H}$ be a sequence of mutually disjoint sets then by the $\sigma$-additivity of the measure $\mathbb{P}(\bigcup_{i=1}^\infty A_i) = \sum_{i=1}^\infty \mathbb{P}(A_i)$ and
        \begin{align*}
            \mathbb{P}(B\cap \left(\bigcup_{i=1}^\infty A_i\right)) &= \mathbb{P}(\bigcup_{i=1}^\infty (B \cap A_i))\\
            &= \sum_{i=1}^\infty \mathbb{P}(B\cap A_i) \quad (\sigma\textrm{-add.})\\
            &= \sum_{i=1}^\infty \mathbb{P}(B) \mathbb{P}(A_i)\\
            &= \mathbb{P}(B)\mathbb{P}(\bigcup_{i=1}^\infty A_i).
        \end{align*}
    \end{enumerate}
    So $\mathcal{H}$ is a $\lambda$-system.

    \noindent Let $\mathcal{G} = \{\{X\geq s\}: s\in \mathbb{R}\}$. It can be shown that $\mathcal{G}$ is a $\pi$-system in a similar way as we did for $\mathcal{E}$ and, as before, $\sigma(\mathcal{G}) \subseteq \sigma(X)$ (by definition of $\sigma(X)$ and $\mathcal{G} \in \mathcal{H}$ (by assumption). Applying the $\pi$-$\lambda$ theorem we obtain
    \begin{align*}
        \sigma(X) \subseteq \sigma(\mathcal{G}) \stackrel{\pi-\lambda}{\subseteq} \mathcal{H} \subseteq \sigma(X),
    \end{align*}
    which implies $\mathcal{H} = \sigma(X)$. Thus now we have
    $$ \mathbb{P}(A\cap B) = \mathbb{P}(A)\mathbb{P}(B) \ \forall A\in \sigma(X), B\in \sigma(Y).  $$
\end{solution}

\begin{solution}[5.4]{Castella, A.}
    \begin{itemize}
        \item We begin by showing that $\max(f,g)$ is measurable. It is clear that for all $t \in \mathbb{R}$ we have
        $$
            \max(f,g) \leq t \implies f \leq t \land g \leq t.
        $$
        Thus let us take such an arbitrary $t$. We find that the equality
        $$
            \{\omega \in \Omega : \max(f,g)(\omega) \leq t\} = \{\omega \in \Omega : f(\omega) \leq t \land g(\omega) \leq t\}
        $$
        directly follows. Additionally, we note that this set is the intersection of the sets
        $$
            \{\omega \in \Omega : f(\omega) \leq t\}
        $$
        and
        $$
            \{\omega \in \Omega : g(\omega) \leq t\}.
        $$
        Since $f$ and $g$ are both measurable functions, we know that the two above mentioned sets are measurable. Additionally, since $\mathcal{F}$ is a $\sigma$-algebra, we know that the intersection of the two sets must be in $\mathcal{F}$ as well. Thus we conclude that the set
        $$
            \{\omega \in \Omega : \max(f, g)(\omega) \leq t\}
        $$
        is indeed measurable. Since $t$ was arbitrary, this holds for all $t \in \mathbb{R}$. Thus we can conclude that $\max(f,g)$ is indeed a measurable function.
        \item We will now prove that $\min(f,g)$ is a measurable function as well. We note that $-f$ and $-g$ are clearly measurable functions, since
        $$
            \{\omega \in \Omega : f(\omega) \leq t\} = \{\omega \in \Omega : -f(\omega) > -t\}
        $$
        holds for all $t \in \mathbb{R}$. The same applies to any measurable function. Thus the measurablility of $-g$ and $-f$ directly follows. By what we proven in the previous item we find that $\max(-f, -g)$ must be measurable. This implies that the function
        $$
            -\max(-f,-g)
        $$
        must be measurable as well. Finally we note that
        $$
            \min(f,g) = -\max(-f,-g),
        $$
        which implies that $\min(f,g)$ is a measurable function.
        \item Since the function $0$ is clearly measurable and $f$ and $g$ were arbitrary measurable functions, we know that
        $$
            -\min(f,0)
        $$
        and
        $$
            \max(f,0)
        $$
        must be measurable as well. Let us now prove an intermediary. We will show that $f+g$ is measurable. We note that for some arbitrary $t \in \mathbb{R}$, we have that $(f+g)(\omega) < t$ if and only if there exists a rational number $r$ such that $f(\omega) < r$ and $g(\omega) < t-r$. Therefore we find that
        $$
            \{\omega \in \Omega : (f+g)(\omega) < t\} = \bigcup_{r\in\mathbb{Q}}\left(\{\omega \in \Omega : f(\omega) < r\}\cap \{\omega \in \Omega : g(\omega) < t-r\}\right).
        $$
        Since the intersection of two measurable sets must be measurable, each term in the union is measurable. Since countable unions of measurable sets must be measurable, we conclude that the set
        $$
            \{\omega \in \Omega : (f+g)(\omega) < t\}
        $$
        is indeed measurable. Since $t$ was arbitrary, we have proven the intermediary that $f+g$ is measurable. Thus we can now use this result to show that
        $$
            \max(f,0) - \min(f,0)
        $$
        is measurable. Since
        $$
            |f| = \max(f,0) - \min(f,0)
        $$
        we conclude that the function $|f|$ is indeed measurable.
    \end{itemize}
\end{solution}

\begin{solution}[5.7]{Castella, A.}
    We will present two different proofs for this problem. The first proof will make us of the hint and properties of measurable sets. The second proof however, will use of the result from Problem 5.8 rather than the hint.
    \begin{itemize}
        \item Let us begin by denoting the set
        $$
            A = \{\omega \in \Omega : \lim_{n\rightarrow\infty}f_n(\omega) = 0\}.
        $$
        We will now show that this set is equivalent to the set denoted by
        $$
            B = \bigcap_{k=1}^\infty\bigcup_{N=1}^\infty\bigcap_{n=N}^\infty\{\omega \in \Omega : |f_n(\omega)| < \frac{1}{m}\}.
        $$
        We will prove this equivalence in the standard way, by choosing an element $\omega$ from either set and showing that it is contained in the other. We split this into two parts.
        \begin{itemize}
            \item Let us take some $\omega \in A$. We note that for all $m \geq 1$, and therefore also all $k \in \mathbb{N}$ (where $k \leq m$), there exists some $N' \in \mathbb{N}$ such that
            $$
                \forall_{n \geq N}: |f_n(\omega)| < \frac{1}{m} \leq \frac{1}{k}.
            $$
            Thus it is clear that, for all $k \in \mathbb{N}$, we have
            $$
                \omega \in \bigcap_{n = N'}^\infty\{\omega \in \Omega : |f_n(\omega)| < \frac{1}{k}\}
            $$
            and therefore also
            $$
                \omega \in \bigcup_{N=1}^\infty\bigcap_{n = N}^\infty\{\omega \in \Omega : |f_n(\omega)| < \frac{1}{k}\}.
            $$
            Since we stated that this holds for all $k \in \mathbb{N}$, it is clear that
            $$
                \omega \in B.
            $$
            We therefore conclude this direction of the proof.
            \item Let us take some $\omega \in B$. From the definition of our set, we can derive that this implies that for all $k \in \mathbb{N}$, there exists $N \in \mathbb{N}$ such that
            $$
                \forall_{n \geq N}: |f_n(\omega)| < \frac{1}{k}.
            $$
            Let us thus take some arbitrary $m > 0$. We know that for all such $m$ there exists some $k \in \mathbb{N}$ such that $k \geq m$ which implies that $\frac{1}{k} \leq \frac{1}{m}$. Since there always exists $N \in \mathbb{N}$ such that the above holds, we find that
            $$
                \forall_{n \geq N}: |f_n(\omega)| < \frac{1}{k} \leq \frac{1}{m}.
            $$
            Following the hint, this implies that
            $$
                \lim_{n\rightarrow\infty}f_n(\omega) = 0.
            $$
            Thus we find that
            $$
                \omega \in A.
            $$
        \end{itemize}
        Having proven both directions, we can conclude that the sets $A$ and $B$ are equal. We can now use this to prove the measurability of $A$ by proving the measurability of $B$ instead. However, in order to do this, we must first prove the measurability of the set
        $$
            \{\omega\in\Omega : |f_n(\omega)| < \frac{1}{k}\}
        $$
        for all $k \in \mathbb{N}$. We begin by noting that from the measurability of $f_n$ and the fact that $\frac{1}{k} \in \mathbb{R}$ it follows that
        $$
            \{\omega \in \Omega : f_n(\omega) < \frac{1}{k}\}
        $$
        and
        $$
            \{\omega \in \Omega : f_n(\omega) > -\frac{1}{k}\}
        $$
        are measurable sets. The intersection of these two sets is equal to
        $$
            \{\omega \in \Omega : |f_n(\omega)| < \frac{1}{k}\}.
        $$
        Since the intersection of two measurable sets must be measurable (as they form a $\sigma$-algebra), we find that this set is measurable. We now note that all countable intersections of measurable sets must be measurable. Thus the set
        $$
            \bigcap_{n = N}^\infty \{\omega\in\Omega : |f_n(\omega)| < \frac{1}{k}\}
        $$
        is measurable for all $N, k \in \mathbb{N}$. By the same countability argument for unions we find that the set
        $$
            \bigcup_{N=1}^\infty\bigcap_{n = N}^\infty \{\omega\in\Omega : |f_n(\omega)| < \frac{1}{k}\}
        $$
        and therefore also the set
        $$
            \bigcap_{k=1}^\infty\bigcup_{N=1}^\infty\bigcap_{n = N}^\infty \{\omega\in\Omega : |f_n(\omega)| < \frac{1}{k}\}
        $$
        must be measurable. Thus we conclude that $B$ is measurable and therefore $A$ must be measurable.
        \item We begin this proof by proving the measurability of the functions $\limsup_{n\rightarrow\infty}f_n$ and $\liminf_{n\rightarrow\infty}f_n$. In order to do this we first prove the measurability of the functions $\sup_{n\geq k}f_n$ and $\inf_{n\geq k}f_n$ for all $k \in \mathbb{N}$. Let us take an arbitrary $t \in \mathbb{R}$. By the definition of the supremum and the infimum we find that
        $$
            \{\omega \in \Omega : \sup_{n \geq k}f_n(\omega) \leq t\} = \bigcap_{n=k}^\infty \{\omega \in \Omega : f_n(\omega) \leq t\}
        $$
        and
        $$
            \{\omega \in \Omega : \inf_{n \geq k}f_n(\omega) \leq t\} = \bigcup_{n=k}^\infty \{\omega \in \Omega : f_n(\omega) \leq t\}.
        $$
        Since $f_n$ is measurable for all $n$, we find that the above sets must be measurable. Since $t$ was arbitrary, we find that this holds for all $t \in \mathbb{R}$ and therefore $\sup_{n \geq k}f_n$ and $\inf_{n \geq k}f_n$ are measurable functions. We note that $k$ was arbitrary as well. Thus let us define the sequences $g_k = \inf_{n \geq k}f_n$ and $h_k = \sup_{n \geq k}f_n$. These are two sequences of measurable sets just like $f_n$ and thus we find that, setting $k = 1$, the functions
        $$
            \inf_kh_k
        $$
        and
        $$
            \sup_kg_k
        $$
        are measurable. Finally, we know that
        $$
            \limsup_nf_n = \inf_n\sup_{k\geq n}f_n
        $$
        and
        $$
            \liminf_nf_n = \sup_n\inf_{k \geq n}f_n,
        $$
        which implies the measurability of the functions $\limsup_nf_n$ and $\liminf_nf_n$. We can now use this result combined with the result from Problem 5.8 to prove the exercise. Using problem 5.8, and the measurability of the 0 function, we find that
        $$
            \{\omega \in \Omega : \liminf_nf_n(\omega) = 0\}
        $$
        and
        $$
            \{\omega \in \Omega : \limsup_nf_n(\omega) = 0\}
        $$
        are measurable sets. Finally, we know that
        $$
            \limsup_{n\rightarrow\infty}f_n(\omega) = \limsup_{n\rightarrow\infty}f_n(\omega) = \lim_{n\rightarrow\infty}f_n(\omega)
        $$
        for all $\omega$ where the limit exists. From this we find that
        $$
            \{\omega \in \Omega : \lim_{n\rightarrow\infty}f_n(\omega) = 0\} = \{\omega \in \Omega : \liminf_{n\rightarrow\infty}f_n(\omega) = 0\}\cap\{\omega \in \Omega : \limsup_{n\rightarrow\infty}f_n(\omega) = 0\}.
        $$
        Since the intersection of measurable sets must be measurable, we now conclude the proof.
    \end{itemize}
\end{solution}

\begin{solution}[5.8]{Castella, A.}
    We begin the proof by showing that the set
    $$
        \{\omega \in \Omega : f(\omega) < g(\omega)\}
    $$
    is measurable. We know that $\omega$ is in this set if and only if there exists some $r \in \mathbb{Q}$ such that
    $$
        f(\omega) < r < g(\omega).
    $$
    Thus we find that
    $$
        \{\omega \in \Omega : f(\omega) < g(\omega)\} = \bigcup_{r \in \mathbb{Q}}\{\omega\in\Omega : f(\omega) < r\} \cap \{\omega\in\Omega : g(\omega) > r\}.
    $$
    Since $\mathbb{Q}$ is countable and both the sets
    $$
        \{\omega \in \Omega : f(\omega) < r\}
    $$
    and
    $$
        \{\omega \in \Omega : g(\omega) > r\}
    $$
    are measurable, we find that the set $\{\omega \in \Omega : f(\omega) < g(\omega)\}$ is measurable as well. Since $g$ and $f$ are arbitrary measurable functions, we now that if we flip the functions, then the new set is still measurable. Additionally, a $\sigma$-algebra contains the complement of all of its sets and thus we find that
    $$
        \{\omega \in \Omega : g(\omega) < f(\omega)\}^c = \{\omega \in \Omega : f(\omega) \leq g(\omega)\}
    $$
    is measurable as well. By the same argument, the set
    $$
        \{\omega \in \Omega : g(\omega) \leq f(\omega)\}
    $$
    is measurable too. Finally, since finite intersections of measurable sets are measurable, we find that the set
    $$
        \{\omega \in \Omega : g(\omega) = f(\omega)\} = \{\omega \in \Omega : f(\omega) \leq g(\omega)\} \cap \{\omega \in \Omega : g(\omega) \leq f(\omega)\}
    $$
    is measurable. We thus conclude the proof.
\end{solution}

\begin{solution}[5.14]{Castella, A.}
    We have already proven in problem 5.4 that if $f$ is measurable, then $\max(f,0)$ and $\min(f,0)$ are measurable as well. Thus we only need to prove the other direction of the implication. Let us assume that $f^+$ and $f^-$ are measurable. Then, by intermediaries proven in problem 5.4, we know that $f^+ + f^-$ is measurable as well. It is clear that
    $$
        f = f^+ + f^-
    $$
    and therefore we can already conclude that $f$ is measurable. Thus we have proven both directions of the implication. 
\end{solution}