\begin{solution}[3.11]{Beerens, L.}
    \begin{itemize}
        \item Let $x_1,x_2\in\mathbb{R}$ with $x_1<x_2$. Then
        $$
            F(x_2) = \mu((-\infty,x_1]\cup (x_1,x_2]) = \mu((-\infty, x_1)) + \mu((x_1,x_2])\geq\mu((-\infty, x_1)) = F(x_1).
        $$
        Therefore, $F$ is non-decreasing.
        
        \item Consider $\lim_{x\downarrow x_0}F(x)$. This can be written as 
        $$
            \lim_{x\downarrow x_0}\mu((-\infty,x]) = \mu((-\infty,x_0]) + \lim_{x\downarrow x_0}\mu((x_0,x]).
        $$
        We shall now prove that $\lim_{x\downarrow x_0}\mu((x_0,x]) = 0$. We know that this limit exists, so we can look at a sequence that converges to $x_0$ from above instead. Choose some $\epsilon>0$. Then 
        $$
            \mu((x_0,x_0+\epsilon]) = \sum_{n=1}^\infty \mu((x_0+\frac{\epsilon}{n+1},x_0+\frac{\epsilon}{n}]).
        $$
        Note that this sum is convergent, which implies that
        $$
            \lim_{i\rightarrow\infty}\sum_{n=i}^\infty \mu((x_0+\frac{\epsilon}{n+1},x_0+\frac{\epsilon}{n}]) = 0.
        $$
        Let $\epsilon_i=\frac{\epsilon}{i}$ for all $i\in\mathbb{N}$. Then 
        $$
            \lim_{x\downarrow x_0}\mu((x_0,x]) = \lim_{i\rightarrow\infty}\mu((x_0,x_0+\epsilon_i]).
        $$
        In addition,
        $$
            \mu((x_0,x_0+\epsilon_i]) = \sum_{n=i}^\infty \mu((x_0+\frac{\epsilon}{n+1},x_0+\frac{\epsilon}{n}]).
        $$
        Therefore, 
        $$
            \lim_{i\rightarrow\infty}\mu((x_0,x_0+\epsilon_i]) = 0.
        $$
        Thus,
        $$
            \lim_{x\downarrow x_0}\mu((-\infty,x]) = \mu((-\infty,x_0]),
        $$
        from which we can conclude that $F$ is right-continuous. 
        
        \item We shall turn towards $\lim_{x\rightarrow -\infty}F(x)$. We know that this limit exists, so we can look at a sequence that diverges to $-\infty$ instead. Choose some arbitrary $x\in\mathbb{R}$. Now,
        $$
            \mu((-\infty,x]) = \mu\left( \bigcup_{n=0}^\infty (x-n-1,x-n] \right) = \sum_{n=0}^\infty\mu((x-n-1,x-n]).
        $$
        Note that the sum is convergent, which implies that
        $$
            \lim_{i\rightarrow\infty}\sum_{n=i}^\infty\mu((x-n-1,x-n]) = 0.
        $$
        Let $x_i = x-i$ for all $i\in\mathbb{N}$. Then 
        $$
            \lim_{x\rightarrow -\infty}F(x) = \lim_{i\rightarrow \infty}F(x_i).
        $$
        In addition
        $$
            \mu((-\infty, x_i]) = \sum_{n=i}^\infty\mu((x-n-1,x-n]).
        $$
        Therefore, 
        $$
            \lim_{i \rightarrow \infty}\mu((-\infty, x_i]) = 0.
        $$
        Thus, we find that 
        $$
            \lim_{x\rightarrow -\infty}F(x)= \lim_{i\rightarrow \infty}F(x_i)=0.
        $$
        
        \item Finally, we consider $\lim_{x\rightarrow \infty}F(x)$. We know that this limit exists, so we can look at a sequence that diverges to $\infty$ instead. Let $(x_n)$ be a sequence in $\mathbb{R}$ defined by $x_n = n$. Then,
        $$
            \lim_{x\rightarrow \infty}F(x) = \lim_{n\rightarrow \infty}F(x_n).
        $$
        In addition, for all $n\in\mathbb{N}$ we have
        $$
            F(x_n) = \sum_{i=-n}^{\infty}\mu((-i-1,-i]).
        $$
        Therefore,
        $$
            \lim_{n\rightarrow\infty}F(x_n) = \sum_{i=-\infty}^{\infty}\mu((-i-1,-i]) = \mu\left( \bigcup_{i=-\infty}^{\infty}(-i-1,-i] \right) = \mu(\mathbb{R}).
        $$
        Thus we can conclude that $\lim_{x\rightarrow \infty}F(x) = \mu(\mathbb{R})$.
    \end{itemize}
\end{solution}